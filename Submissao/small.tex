\documentclass[12pt,english]{amsart}
%\usepackage[margin=0.5in]{geometry}
\usepackage[a4paper, margin=2cm]{geometry}
\linespread{1.5}

%\usepackage{fontspec}
%\setmainfont{Times New Roman}
\usepackage{mathptmx}

\usepackage{graphicx}
\usepackage{wrapfig}
\usepackage{lscape}
\usepackage{rotating}
\usepackage{epstopdf}

\usepackage{verbatim}
\usepackage{url}
\usepackage{xspace}
%\usepackage{wrapfig}

\usepackage{listings}
\usepackage{color}
    \definecolor{light}{gray}{0.97}
    \definecolor{dark}{gray}{0.30}
\lstset{
%columns=fullflexible,
%basicstyle=\ttfamily,
escapeinside={||},
    %mathescape=true,
    language=C, % choose the language of the code
    basicstyle=\fontfamily{pcr}\selectfont\scriptsize\color{black},
    keywordstyle=\color{black}\bfseries, % style for keywords
    numbers=none, % where to put the line-numbers
    numberstyle=\tiny, % the size of the fonts that are used for the line-numbers
    backgroundcolor=\color{light},
    showspaces=false, % show spaces adding particular underscores
    showstringspaces=false, % underline spaces within strings
    showtabs=false, % show tabs within strings adding particular underscores
    %frame=single, % adds a frame around the code
    tabsize=2, % sets default tabsize to 2 spaces
    %rulesepcolor=\color{gray}
    captionpos=b, % sets the caption-position to bottom
    breaklines=false, % sets automatic line breaking
    %breakatwhitespace=false,
    numbersep=2em,
    % C was used in the blocksworld example to refer to block C and nowhere else
    emph={par,or,hor,do,end,loop,code,await,pause,emit,input,event,call,with,%
          var,and,then,else,return,pure,deterministic,nohold,finalize,%
          class, every, FOREVER, this, spawn, in, pool, watching, until, 
          interface, each, abort, when, signal, PROC, CHAN, SIGNAL, PAR, not,
          bool, tag, escape, traverse,implementation,output,true,false,
          native,@const,@pure,@safe,define,public,private,none},
    emphstyle={\bfseries},
    commentstyle=\color{dark}\scriptsize,
    %xleftmargin=20pt,
    %xrightmargin=20pt,
    framesep=20pt,
    %upquote=true,
    %aboveskip={1.5\baselineskip},
}

%\newcommand{\CEU}{\textsc{C\'{e}u}\xspace}
\newcommand{\code}[1] {{\small{\texttt{#1}}}}

\usepackage{enumitem}
\setlist{nolistsep}

% Auto-Standby
\title{Energy Efficiency for IoT Software in the Large
        }%\\ \large{Research project proposal submitted to the Instituto Serrapilheira}}

%\author{Anonymous Author}

\begin{document}

\date{}
\maketitle

\vspace{-1cm}
\begin{comment}
\abstract{
TODO
}
\end{comment}

%\newpage
\section{Professional Narrative}

I work in the field of \emph{Programming Languages} with a focus on
\emph{Real-Time Reactive Systems}.
Reactive systems interact continuously with the external world through sensors
and actuators (e.g., buttons, LEDs, and motors).
When these interactions have to comply with strict deadlines, the system is
said to be \emph{real time}.
Most everyday applications, such as office suites, phone apps, and video games
are reactive but fairly tolerant to delays.
In contrast, embedded real-time systems, such as medical appliances,
avionic systems, and IoT devices, must conform to stricter deadlines.

%Since the beginning of my Masters at PUC-Rio, I have been working with
%programming languages support for reactive systems.
In the beginning of my Masters, I joined a group supporting
\emph{Ginga}, the software standard of the Brazilian Digital TV
System~\cite{ncl.abnt}, which also became an UN's ITU
standard~\cite{ncl.itu}.
%In a digital system, broadcasters can transmit and execute software in real
%time along with the primary audio-visual content.
% in order to enrich the
%audience experience (e.g., focused content, online shopping, etc.).
During this period, I worked on a non-intrusive integration of the two
standardized languages NCL and Lua through a reactive programming
interface.
%I have also participated in standardization meetings with broadcasters, TV
%manufacturers, government, and academia.
The outcome of this work was published in the \emph{ACM DocEng} in 2009.
I also wrote the chapter on developing with Lua and NCL in Ginga's main
reference book. % on developing Digital TV applications.

\begin{comment}
NCL is a declarative language to specify how multiple media objects are
composed together in cause-effect relationships to form a complete multimedia
presentation (e.g., after 1 minute of a video playback, show an image for 10
seconds).
Lua is a scripting language that extends the power of NCL with generic
programming language constructs such as loops, variables, and arithmetic
expressions.
During the same period, I wrote my Masters thesis on the design of
\emph{LuaGravity}, a programming language that extends Lua with support to
handle events from the environment.
Programs written in LuaGravity behave like spreadsheets, in which a user
event (e.g., typing a new number in a cell) propagates throughout the program
and changes all of its dependencies automatically.
\end{comment}

In 2009, I started the PhD program with the goal of designing a new
reactive language from scratch, now targeting resource-constrained embedded
systems.
On the one hand, these platforms are typically six orders of magnitude less
powerful than off-the-shelf personal computers in terms of memory size and CPU
speed.
On the other hand, embedded applications need higher reliability to operate
without human supervision for long periods of time.
Furthermore, a significant subset of embedded systems is deployed in locations
without power lines and need to run on batteries.
Hence, a programming language targeting constrained embedded systems needs to
be small, reliable, and energy efficient.
At the end of 2012, I earned a scholarship from SAAB Aerospace for a six-month
research period. % in the field of \emph{Secure Wireless Sensor Networks}.
I joined an University in Sweden and
we rewrote industrial-grade network drivers and protocols using my
research language and attested a considerable reduction in source code size
with similar resource usage in comparison to C. % of memory and CPU.
In 2013, we published a paper with these results in the \emph{ACM SenSys}.

After graduating, I contributed to the
project \emph{Cidades Inteligentes} (Smart Cities), which involves 20 
Universities and aims to build an infrastructure for the IoT.
%
In 2015, in cooperation with a PhD student, we published a paper the
\emph{ACM TOSN}, showing that we can reprogram embedded devices remotely using
less energy than the state of the art.
%
During the same period, I published a paper in the \emph{Int'l Conference on Modularity}
proposing a new concurrency abstraction that leads to more modular
programs.
%
I was also invited to present my research language in the annual
meeting of the \emph{Working Group on Language Design}, which is affiliated
with UNESCO's IFIP.%\footnote{IFIP: \url{http://www.ifip.org/}}.

Currently, as an associate professor at a Brazilian University,
%In 2016, I became an associate professor (\emph{professor adjunto}) at UERJ.
I continue to investigate how the design of languages can affect
the development of reactive applications.
For the last 4 years, I have have been either publishing in or acting in the
committee of the \emph{Int'l Workshop on Reactive Languages}.
In 2017, I published a consolidating paper on the design of
my research language in the \emph{ACM TECS}.

% multimedia, games, sensor networks, embedded systems

%- mobile phone game industry
%- empresa americana
%- Future of Programming
%- OpenSource
%- GSoC

%\newpage
\section{Science Outreach: The Role of IoT Software in Energy Efficiency}

According to the International Energy Agency (IEA)~\cite{iea.data}, there were
around 14 billion traditional network-connected devices in 2013 (e.g., mobile
phones and smart TVs).
This number is expected to increase to 50 billion by 2020 with
the proliferation of IoT devices (e.g., smart bulbs and fitness wearables).
%
IoT and traditional network-connected devices already outnumber people on the
planet by a factor of two, and the amount of data traffic is expected to grow
at an exponential rate in the next years.
%By 2050, the energy demands of networked devices should exceed the current
%consumption of Germany and Canada combined, corresponding to 6\% of current
%global electricity consumption.
%
However, most of the energy to power these devices is consumed while they are
in \emph{standby mode}.
%, i.e., when their software is neither transmitting or processing data.
%
The annual $CO_2$ emissions related to standby in Australia are equivalent to
those of 1 million cars.
%
%This undesirable inefficiency opens opportunities for improvements in energy
%savings closer to 65\% only considering currently available technologies.
%
The projected growth of IoT devices together with the surprising effects of
standby consumption, made network standby efficiency one of the six
pillars of G20's \emph{Energy Efficiency Action Plan}%
\footnote{G20's Energy Efficiency Action Plan: \url{https://www.iea-4e.org/projects/g20}}.

Other organizations have also reported the importance of energy savings for networked devices.
%
For the Internet Engineering Task Force (IETF), ``energy management is becoming an
additional requirement for networks due to several factors including the rising
energy costs, the increased awareness of the ecological
impact of operating networks and devices, and the regulation of
energy''~\cite{ietf.eman}.
%
For the American Council for an Energy-Efficient Economy (ACEEE),
``the potential for new energy efficiency remains enormous, (...) we must take
a systems-based approach to scale up energy efficiency.
%Intelligent efficiency is a systems-based approach to efficiency that can help
%to meet this need.
(...) intelligent efficiency is adaptive, anticipatory, and networked''~\cite{aceee.1}.
%
%The pioneer \emph{ENERGY STAR} program~\cite{energystar} states that
%``networked devices and networking equipment, as integrated systems, have the
%potential to contribute significant net energy savings''.
% \footnote{ENERGY STAR program \url{https://www.energystar.gov/index.cfm?c=prod_development.prod_development_epa_workshop}}
%
%The US Environmental Protection Agency (EPA) and Information Technology
%Industry Council (ITI) held a workshop to explore roles for the \emph{ENERGY
%STAR} and others program in promoting savings through intelligent energy
%efficiency~\cite{aceee.2}.
%, which differs from individualized
%component-based energy efficiency in that it is adaptive, anticipatory, and
%networked.

With regard to concrete initiatives, IEA's \emph{Electronic Devices and Networks}
focuses specifically on the issue of networked device standby%
\footnote{EDNA initiative: \url{https://edna.iea-4e.org/}}.
%It is one of the six fronts of G20's \emph{Energy Efficiency Action Plan}%
%\footnote{\url{https://www.iea-4e.org/projects/g20}}.
%Network standby refers to the minimum power a device requires to maintain its
%connection when not communicating or computing actively.
ACEEE's \emph{Intelligent Efficiency} promotes a systematic approach
to optimize the behavior of cooperating devices in order to achieve energy
savings as a whole.
Both approaches, device and system based, involve mostly software solutions,
since energy savings are dynamic policies that depend on the application
demands and device battery levels at a given moment in time.
%
There are also low-power standards for the IoT
infrastructure that suit different needs of range, throughput, and physical
distribution~\cite{iot.energy.2}.
For instance, \emph{Bluetooth Low-Energy (BLE)} is a replacement for classic
Bluetooth and is designed for lower data throughput in personal area
networks (PANs).
\emph{6LoWPAN} adapts the IPv6 internet standard to low-power and
limited-processing devices.
%
These technologies make radio transmissions more efficient,
support flexible network topologies, and reduce data traffic considerably.
They also enable sleep modes that cut energy consumption to minimum levels.
%
However, these technologies ultimately require software to control their
functionalities and to wisely switch between standby and active modes in order
to build an energy-efficient IoT.

%\newpage
\section{Scientific Summary}

Effective use of low-power standby will play a fundamental role in energy
efficiency for the expected 50 billion IoT devices by 2020~\cite{iea.data}.
%In order to succeed in this challenge, new solutions have to scale to the
%forthcoming mass of IoT software that must use standby modes effectively.
%
This research project aims to address the software challenges, as determined by
IEA, towards an energy-efficient IoT~\cite{iea.data}:
    to ensure that devices employ the lowest possible modes of standby, and
    that devices remain in longest possible periods of standby.
%
\begin{comment}
    \item To ensure that devices employ the lowest possible modes of standby consumption.
    \item To ensure that devices remain in longest possible periods of standby time.
\end{comment}

Given the projected scale of the IoT and the role of low-power standby towards
energy efficiency, this research project has the following goals:

\begin{enumerate}
    \item Address energy efficiency through extensive use of standby.
    \item Target constrained embedded architectures that form the IoT.
    \item Provide standby mechanisms at the programming language level that scale to all applications.
    \item Support transparent/non-intrusive standby mechanisms that reduce barriers of adoption.
\end{enumerate}

% TODO: goal 1 and 3 are also the limitations: only standby, not holistic energy savings

%\goodbreak\noindent
Many energy-aware languages, extensions, and operating systems have been
proposed recently.
%
Some proposals adjust QoS (quality of service) to reduce power consumption~\cite{os.ecosystem,lang.green,lang.enerj,lang.greenweb}.
%(e.g., accuracy of computations)
%os.odyssey.2,os.jouleguard,lang.lab,lang.flexjava,
%
Other proposals offer mechanisms to switch behaviors depending on application
demands and battery levels~\cite{lang.eon,lang.energytypes,lang.gradual,lang.ent}.
%(e.g., disable some functionalities)
%
None of these initiatives take advantage of standby modes
when idle,
but only adapt or eliminate computations while in active modes (not
addressing goal 1).
%
%There are also specialized network protocols that make devices remain in
%low-power standby modes for longer periods of time~\cite{wsn.energy}.
%
%However, protocol-based initiatives only apply to the networked parts of
%applications and have to be programmed carefully to take
%advantage of standby modes (not addressing goals 3 and 4).
%(which is typically the most power hungry subsystem),
%
%This proposal is also unique in the aim to provide \emph{transparent
%mechanisms} such that there are no explicit programming efforts to make
%applications more energy efficient (goal 4 above).

I have been working in the design of a new reactive programming language
targeting resource-constrained embedded systems for the past 8 years.
%
The language is grounded on the synchronous concurrency model,
%which has been
%successfully adopted in the context of hard real-time systems such as avionics
%and automobiles industry since the 80's~\cite{rp.twelve}.
%
%The synchronous model
which trades power for reliability and has a simpler model
of time that suits most requirements of IoT applications.
%
%On the one hand, this model cannot directly express time-consuming
%computations, such as compression and cryptography algorithms, which are
%typically either absent or delegated to auxiliary chips in the context of the
%IoT.
%
%On the other hand,
In this model, all reactions to the external world are guaranteed to be
computed in bounded
time, ensuring that applications always reach an idle state amenable to standby
mode.
%
%Furthermore, reactions to events are part of the vocabulary of the language.
%This way, when an application reaches an idle state, the language has precise
%information about which events can awake that application.
%This not only allows the application to enter standby mode, but effectively use
%the deepest sleeping level considering all possible awaking events.
%While in standby, only a hardware interrupt associated with the events can
%awake the application, making it sleep for longest possible periods of time.
%Despite the advances in energy-aware languages, embedded systems and are still
%primarily programmed in the C language~\cite{es.c}.
%
%This proposal lies at the bottom of the software development
%layers---transparent programming language mechanisms---meaning that \emph{all}
%applications would take advantage of low-power standby modes automatically,
%without extra programming efforts.
%
We expect that existing energy-unaware applications will benefit from savings
in the order of 50\% based on IEA's estimates considering current standby
technologies~\cite{iea.data}, and previous third-party work on transparent energy
awareness~\cite{wsn.tos.2}.

%\newpage
\section{Scientific Project}

\subsection{First Year}
\label{sec.first}

We will require a budget of R\$70k:
R\$60k to support three students, R\$5k for equipment, and
R\$5k for publication and travel costs.
The details are described in the long proposal.

\subsubsection{IoT Hardware Infrastructure}

%The IoT infrastructure is built on top of embedded systems such as sensor
%nodes, network routers, and other low-power microcontroller-based devices and
%appliances.
%
We will use off-the-shelf Arduinos~\cite{arduino} as the main hardware IoT
platform.
Most Arduinos are based on low-cost microcontrollers from
Atmel, such as the \emph{ATmega328p}~\cite{arduino.atmega328p}, supporting
six sleeping modes that can reduce power consumption to very low levels.

%Arduino is also the most popular embedded platform worldwide.
%
    In the academia, there is a lot of research using Arduino in the context of
    the IoT~\cite{arduino.infra,arduino.health,arduino.home,arduino.energy}.
% (e.g., infrastructure~\cite{arduino.infra},
    %healthcare~\cite{arduino.health}, home automation~\cite{arduino.home}, and
    %energy awareness~\cite{arduino.energy}).
    %In addition to the availability of support literature,
    The popularity of
    Arduino will make our own research more accessible and reproducible to
    other groups.
%
    In education, many University courses use Arduino~\cite{arduino.edu.1,arduino.edu.2,arduino.edu.3,arduino.edu.4}.
    We have also been using Arduino in an undergraduate course for the
    past 3 years,
    %Arduino lowers the learning barrier for students, and
    which will allow us to evaluate
    our results with less experienced embedded programmers.
%
    In the hobbyists community, there is an abundance of publicly available software
    which we can adapt to our
    language to evaluate energy efficiency gains.
%
    %Among non-programmers, Arduino is also becoming popular and a lot of
    %software will be produced by them in the near future.
    %We are in the process of creating an interdisciplinary course targeting
    %non-specialists (e.g., arts, design, architecture).
    %This audience will be a good target to evaluate our transparent approach for
    %energy efficiency.

\subsubsection{IoT Software Infrastructure}
\label{sec.method.software}

The typical way to interact with the external world in Arduino is through
polling, which samples an
external device actively to detect changes on its state. % to take an action.
Polling wastes CPU cycles and prevents the device to enter in standby.
In Arduino, even basic functionality, such as timers, A/D converters, and
SPI, uses active polling loops that waste energy in active mode.

In order to provide automatic standby, applications need to be entirely
reactive to events.
%
Recently, we have added support for \emph{interrupt service routines (ISRs)} as
a primitive concept of our language, which will allow us to rebuild the IoT
software infrastructure with standby awareness from the ground up.
%
This process will consist primarily of rewriting device drivers, which are the
pieces of software that interface directly with the hardware.
%
%More concretely, at the lowest level, energy-efficient software depends on
%\emph{interrupt service routines (ISRs)}, which awake the CPU from standby when
%an external peripheral event occurs.
%
This approach will not affect how applications are written at higher levels,
which will remain similar to Arduino applications.
However, instead of wasting cycles in busy-wait loops, the applications will
enter the deepest possible standby mode when idle.

\subsubsection{IoT Applications}
\label{sec.method.apps}

In order to evaluate the gains in energy efficiency with the proposed
infrastructure, we will need to evaluate the consumption of realistic applications.
%
The Arduino community has an abundance of open-source projects which can be
rewritten in our language to take advantage of transparent standby modes.
%
%Also, most peripherals for Arduino ship with drivers and applications that
%explore their functionalities (e.g., accelerometers, touch screens, GPS, etc.).
%
Then, we will be able to compare the original and rewritten versions in terms of energy
consumption to draw conclusions about the effectiveness of transparent standby
modes.
%
The most realistic scenarios for the IoT use radio communication extensively.
%
In this context, we will evaluate from simple handmade ad-hoc protocols to more
complex energy-aware protocols and see to what extent our proposal would
effectively contribute to energy savings.
%
%The \emph{RadioHead} project%
%\footnote{RadioHead project: \url{http://www.airspayce.com/mikem/arduino/RadioHead/}}
%supports dozens of RF (radio frequency) chipsets and provides examples that
%explore ad-hoc mesh as well as low-power energy modes.
%
%We plan to support at least two chipsets, such as \emph{nRF24L01}~\cite{radio.nrf24l01}
%and \emph{RFM69HCW}~\cite{radio.rfm69hcw}, both of which have good support for
%low power consumption.
%
In the long term, we expect to make the case for
developers to rewrite their applications in our language to take advantage of
standby modes for free.
%
%Towards this direction, we will evaluate rewritten IoT applications based on the
%following criteria:
Towards this direction, we will evaluate the time it takes to rewrite an application and
the real gains in energy efficiency.
%
\begin{comment}
\item[Time to rewrite:]
This criteria relates to the incentives to rewrite existing applications to
our language.
It is a tradeoff between the expected rewriting times and gains in energy
efficiency as compared to standard Arduino code.
\item[Coding ``aesthetics'':]
To lower the adoption barrier, it is also important that the proposed
programming style is sufficiently familiar, and, at least, as easy to write applications.
\item[Energy consumption:]
More importantly, rewritten applications should have significant gains in
energy efficiency to justify a complete application rewrite.
\end{comment}

\subsubsection{\textbf{Expected Contributions}}

%\begin{enumerate}

%\item \textbf{An environment-aware system language:}
    The dedicated vocabulary of our language for events raises the
    programming abstraction to a level closer to the domain of IoT, providing
    more safety and expressiveness to programmers.
    As far as we know, ISR support at the language level has not been
    tried in the past.
    %The novel support for ISRs would complete the whole development cycle, from
    %the application down to the hardware, without operating system support.

%\item \textbf{An energy-aware system language:}
    Our proposal aims to make all applications subject to standby modes
    transparently.
    Being part of the software infrastructure, only device drivers will require
    explicit energy management, and all applications built on top of them will
    benefit from energy efficient automatically.

%\item \textbf{A realistic programming alternative for the IoT in the Arduino:}
    Our language is a 8-year project and has a mature open-source
    implementation that is publicly available.
    With the adaptation to the context of energy-efficient IoT, the language
    may become a practical alternative for Arduino in the short term.

%\item \textbf{Effective energy-efficient IoT applications:}
    %We expect substantial gains in energy efficiency for all applications that
    %do not manage energy explicitly.
    %Most Arduino APIs use busy-wait polling to wait for input, which are very
    %energy inefficient, and could take advantage of transparent standby.
    %We also expect gains in comparison to some ad-hoc energy-aware
    %implementations---we believe that, for the most cases, a transparent
    %solution is more effective than explicit programming efforts.
%\end{enumerate}

%\newpage
\subsection{The Three Years to Follow}

\subsubsection{Systematic Energy Awareness}
\label{sec.method.systematic}

Systematic energy awareness
considers the IoT network as a whole, with cooperating devices trying to
maximize energy efficiency globally.
%
In this context, \emph{adaptive computing}~\cite{adaptive} is the capability of
the IoT to adapt energy consumption dynamically, based on application demands
and levels of power supply during execution.
%
Ultimately, the goal is still that each device maximizes its idle
periods to be more susceptible to standby modes.
Hence, the research of the
first year is a prerequisite for this phase.
%
We will continue to investigate how language mechanisms can aid energy
efficiency, but now in the context of adaptive computing.

The research method for systematic energy awareness will be similar to the
one adopted for device-based awareness (as discussed in
Section~\ref{sec.method.apps}):
take existing energy-aware IoT protocols and applications for Arduino, rewrite
them using non-intrusive mechanisms, and use the same evaluation criteria
(i.e., time to rewrite, coding aesthetics, and energy consumption).

\subsubsection{Complex IoT Architectures and Applications}
\label{sec.method.complex}

%The niche of constrained embedded systems, which includes Arduino, covers the
%substantial (and increasing) portion of IoT applications.
%They typically do not require significant computing resources but are sensitive
%to physical dimensions and energy consumption.
%
%However,
The IoT also consists of more traditional networked devices, such as
routers, servers, and smartphones.
%
As of 2016, there were 3.9 billion smartphone subscriptions worldwide, and
this number is expected to reach 6.8 billions by 2022~\cite{ericsson.mobility}.

Smartphones have similar restrictions in battery consumption and could also
take advantage of the techniques we propose for constrained embedded systems.
%
In order to transpose the barrier from constrained devices to smartphones for
the IoT, we will take a similar path as presented in Section~\ref{sec.first}:
%
\begin{description}
\item[Hardware Infrastructure]
We will use the BeagleBone Black~\cite{bbb.manual}, which shares similar goals with
Arduino, providing a cheap and open-source platform but which is suitable for rich
applications, such as graphical user interfaces, multimedia, and games.
\item[Software Infrastructure]
In order to ensure automatic standby for applications, all software
infrastructure, mostly device drivers, needs to be rebuilt from scratch using
ISRs in our language.
\item[Applications]
In addition to IoT applications, typical smartphone applications, such as
instant messaging and internet browsing, can also maximize energy efficiency
through standby.
We will rewrite from simple applications, such as a graphical clock, to more
complex networked applications, such as a web browser, and evaluate their energy consumption.
\end{description}

\subsubsection{\textbf{Expected Contributions}}

%\begin{enumerate}

%\item \textbf{Adaptive and energy-efficient IoT applications:}
    We expect that systematic energy awareness will take advantage of the
    non-intrusive mechanisms of our language to be implemented at a higher
    level of abstraction, resulting in energy-efficient applications that
    require less programming efforts.

%\item \textbf{Energy-efficient smartphones:}
    The transition from simple mobile phones to smartphones resulted in a
    degrading effect on battery lives.
    %Phones used to have autonomy of weeks and now require daily recharges.
    Most of the time, however, smartphones are sitting in our pockets wasting
    energy.
    We expect to increase the battery autonomy considerably while keeping the
    functionality of a modern smartphone.
%\end{enumerate}

\begin{comment}
%\newpage
\section{CONCLUSION}

- story
    - environment awareness
    - boot up a language
    - test in the environment
    - energy awareness
    - ISRs
    - test in the environment
    - OS
    - C is 45 years

TODO

- conslidated research
- some steps already done
- we need human resources

realistic research
    - academic results
    - but also an open-source publicly available language that people can download and use
        - already available
        - all publics: research, hobbyists, students

TODO: long-term research
    - 8 years in the same project
\end{comment}

\newpage
\bibliographystyle{abbrv}
\bibliography{serra}
\end{document}
