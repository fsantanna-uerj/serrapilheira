\vspace{-0.5cm}
\section{Introduction}
\label{sec.introduction}

% TODO-FULL: voltar c/ artigo completo
\begin{comment}
According to the International Energy Agency (IEA), the number of
network-connected devices is expected to reach 50 billion by 2020 with the
expansion of the Internet of Things (IoT)~\cite{iea.data}.
%
However, most of the energy to power these devices will be consumed in
\emph{standby mode}, i.e., when they are neither transmitting or processing
data.
%
For instance, standby power currently accounts for approximately 10--15\% of
residential electricity consumption, and $CO_2$ emissions related to standby
are equivalent to those of 1 million cars~\cite{iea.data,standby.australia}.
%
The projected growth of IoT devices, together with the surprising effects of
standby consumption, made network standby efficiency one of the six
pillars of the \emph{G20's Energy Efficiency Action Plan}%
\footnote{G20's Energy Efficiency Action Plan: \url{https://www.iea-4e.org/projects/g20}}.
%
However, making effective use of standby requires software-related efforts in
order to detect idle periods of activity in a device, identify peripherals
that must remain functional, and apply appropriate sleep mode levels in its
microcontroller.

Given the projected scale of the IoT, the role of low-power standby towards
energy efficiency, and the posed software-related challenges, our research has
the following goals:
\end{comment}
%%%%
According to the International Energy Agency (IEA), the number of
network-connected devices is expected to reach 50 billion by 2020 with the
expansion of the Internet of Things (IoT)~\cite{iea.data}.
%
Most of the energy to power these devices will be consumed in
\emph{standby mode}, i.e., when they are neither transmitting or processing
data.
%
However, making effective use of standby requires software-related efforts in
order to detect idle periods of activity in a device, identify peripherals
that must remain functional, and apply appropriate sleep mode levels in its
microcontroller.
Therefore, our research has the following goals:
%%%%
%
%\begin{enumerate}
(i)   address energy efficiency through rigorous use of standby;
(ii)  target low-power, resource-constrained embedded architectures that form
      the IoT;
(iii) provide standby mechanisms at the programming language level that scale
      to all applications; and
(iv)  support transparent/non-intrusive standby mechanisms that reduce barriers
      of adoption.
%\end{enumerate}

Our proposal lies at the bottom of the software development
layers---programming language mechanisms---meaning that \emph{all} applications
should take advantage of low-power standby modes automatically, without extra
programming efforts.
%
We extend the programming language \CEU~\cite{ceu.sensys13,ceu.tecs17} with
support for interrupt service routines (ISRs) and with a simple power
management runtime (PMR).
%
In contrast with other concurrency models (e.g., thread and actor based), the
synchronous semantics of \CEU guarantees that reactions to the environment
always reach an idle state amenable to standby.
%
Previous work~\cite{ceu.sensys13} demonstrates the expressiveness of \CEU in
the context of Wireless Sensor Networks and discusses the development of
drivers, network protocols, and full applications in the language.
It also attests a small overhead of memory in comparison to C (around 5-10\%),
thus being a suitable alternative for constrained devices.
The language runtime is in the order of a few kilobytes only (less than 5Kb).

In our approach, each supported microcontroller requires hooks in C for the
ISRs and PMR, and each peripheral requires a driver in \CEU.
These are write-once code that is typically packaged and distributed in a
software development kit (SDK).
%
Then, all new applications built on top of these drivers take advantage of
standby automatically.
%
As a proof of concept, we provide an open source SDK %
%\footnote{https://github.com/fsantanna/ceu-arduino/}
with support for 8-bit
\emph{AVR/ATmega} and 32-bit \emph{ARM/Cortex-M0} microcontrollers, and a
variety of peripherals, such as for GPIO, A/D converter, USART, SPI, and the
nRF24L01 transceiver.
%
We developed a number of simple applications using these peripherals
concurrently and could verify that they remain in the deepest standby modes for
the longest periods of time.

%Even though the examples in this paper use standby effectively, they are too
%simple to draw any conclusions about power savings in real-world applications.
%That said, our approach is optimal in the sense that the language ensures idle
%states and the runtime can always figure out the most efficient sleeping mode
%possible.
%Nonetheless, this is a work in progress and we acknowledge the importance of a
%quantitative evaluation in the conclusion of the paper as future work.

In Section~\ref{sec.ceu}, we compare the structure of programs in \CEU and
Arduino~\cite{arduino.book}, whose primary goal is to reduce the programming
barrier of adoption for a non-technical audience. % TODO-FULL (e.g., designers and artists).
We show that we can keep the intended sequential reasoning of Arduino even when
applications require non-trivial concurrent behavior.
%
In Section~\ref{sec.standby}, we discuss the software infrastructure that
allows for unmodified programs in \CEU to take advantage of standby
automatically.
%
In Section~\ref{sec.conclusion}, we discuss future work and conclude the paper.

% - standby as well as enabled/disabled, powered, switched
% - application on IoT
% - results
%   - atmega 328p/2560
%   - arm cortex-m0
%   - how much efficiency?
% - limitations

%In Section~\ref{sec.ceu}, we introduce the structured synchronous model and the
%programming language \CEU.
%In Section~\ref{sec.apps}, we evaluate .
%In Section~\ref{sec.ext}, we present the language extensions that support
%transparent standby.

\section{The Structured Synchronous Programming Language \CEU}
\label{sec.ceu}

\begin{comment}
- structured programming
- lexical scope

In summary:

Reactive: code executes in reactions to events

Synchronous: reactions run to completion, i.e., there's no implicit preemption or real parallelism (this avoids explicit synchronization: locks, queues, etc)

Structured: programs use structured control mechanisms, such as "await" (to suspend a line of execution), and "par" (to combine multiple awaiting lines of execution)

Structured programming avoids deep nesting of callbacks letting you write programs in direct/sequential style. In addition, when a line of execution is aborted, all allocated resources are safely released.

In comparison to FRP/dataflow, it is more imperative supporting sequences/loops/conditionals/parallels. The notion of (multiple) program counter is explicit. Also, everything is lexically scoped, there's no GC involved.

In comparison to promises/futures, it provides lexical parallel constructs, allowing the branches to share local variables and, more importantly, supporting safe abortion of code (with the "par/or").
\end{comment}

\CEU is a Esterel-based~\cite{ceu.tecs17} reactive programming language
targeting resource-constrained embedded systems~\cite{ceu.sensys13}.
%
It is grounded on the synchronous concurrency model, which has been
successfully adopted in the context of hard real-time systems such as avionics
and automobiles industry since the 80's~\cite{rp.twelve}.
%
The synchronous model trades power for reliability and has a simpler model
of time that suits most requirements of IoT applications.
%
On the one hand, this model cannot directly express time-consuming
computations, such as compression and cryptography algorithms, which are
typically either absent or delegated to auxiliary chips in the context of the
IoT.
%
On the other hand, all reactions to the external environment are guaranteed to
be computed in bounded time~\cite{ceu.sensys13}, ensuring that applications
always reach an idle state amenable to standby mode.
%
Overall, \CEU aims to offer a concurrent, safe, and expressive alternative to C
with the characteristics that follow:
%
\begin{description}
\item [Reactive:] code only executes in reactions to events and is idle most of
    the time.
\item [Structured:] programs use structured control mechanisms, such as
    \code{await} (to suspend a line of execution), and \code{par} (to combine
    multiple lines of execution).
\item [Synchronous:] reactions run atomically and to completion on each line of
    execution, i.e., there's no implicit preemption or real parallelism.
\end{description}
%
Structured reactive programming lets developers write code in direct style,
recovering from the inversion of control imposed by event-driven
execution~\cite{rp.deprecating,rp.rescala,sync_async.cooperative}.

\subsection{A Motivating Example}
\label{sec.ceu.example}

{\linespread{1}
\begin{figure}[t]
\vspace{2mm}
\begin{minipage}[t]{0.49\linewidth}
\begin{lstlisting}[xrightmargin=0.5cm]
while (1) {
  delay(1000);
  int v =
    analogRead();
  radioWrite(v);
}
\end{lstlisting}
\centering\small{[a] Version in Arduino}
\end{minipage}
%
\begin{minipage}[t]{0.49\linewidth}
\begin{lstlisting}[numbers=left,xleftmargin=-0.2cm]
loop do
  await 1s;
  var int v =
    await AnalogRead();
  await RadioWrite(v);
end
\end{lstlisting}
\centering\small{[b] Version in \CEU}
\end{minipage}
%\rule{8.4cm}{0.37pt}
\caption{ Sequence of I/O operations running in a loop.
\label{lst.direct}
}
\end{figure}
}

Figure~\ref{lst.direct}.a shows a straightforward, easy-to-read code snippet
in Arduino that executes forever in a loop a sequence of operations as follows:
    waits for 1 second (ln. 2),
    performs an A/D conversion (ln. 3--4), and
    broadcasts the read value (ln. 5).
%
Figure~\ref{lst.direct}.b shows the same code in \CEU, with the noteworthy
difference that operations that interact with the environment and take time use
the \code{await} keyword. % and can be easily identified in the program.
%
The traditional structured paradigm encouraged in Arduino (with blocks, loops,
and sequences) allows for simple and readable code, avoiding the complexity of
dealing with ISRs.
%
However, the use of blocking operations, such as \code{delay(1000)} (ln. 2),
prevents that other operations execute concurrently.

\begin{figure}[t]
\begin{minipage}[t]{0.49\linewidth}
\begin{lstlisting}[xrightmargin=0.5cm]
uint32_t prv =
  millis();
while (1) {
  if (radioAvail()) {
    break;
  }
  uint32_t cur =
    millis();
  if (cur>prv+1000) {
    prv = cur;
    int v =
      analogRead();
    radioWrite(v);
  }
}
\end{lstlisting}
\centering\small{[a] Version in Arduino}
\end{minipage}
%
\begin{minipage}[t]{0.49\linewidth}
\begin{lstlisting}[numbers=left,xleftmargin=-0.2cm]
par/or do
  await RadioAvail();
with
  loop do
    await 1s;
    var int v =
      await AnalogRead();
    await RadioWrite(v);
  end
end




.
\end{lstlisting}
\centering\small{[b] Version in \CEU}
\end{minipage}
%\rule{8.4cm}{0.37pt}
\caption{ Achieving concurrency between I/O operations.
\label{lst.inversion}
}
\end{figure}

Suppose that we now want to immediately abort the loop in
Figure~\ref{lst.direct}.a at any time, as soon as a radio message arrives.
%
Since the message might arrive concurrently with any of the blocking
operations, we need to modify the structure of the program in Arduino.
%
Figure~\ref{lst.inversion}.a replaces the blocking \code{delay} to the polling
\code{millis}, which immediately returns the number of milliseconds since the
reset.
Now, we start by registering the current time (ln. 1--2) and, on each loop
iteration, we recheck the time to see if one second has elapsed (ln. 7--9).
Since these operations are non-blocking, we can intercalate their execution
with checks for message arrivals (ln. 4--6).
If the time is up, we start counting it again (ln. 10) before proceeding to the
original operations in sequence (ln. 11--13).
%
The original structured style in Figure~\ref{lst.direct}.a has been drastically
violated to accommodate concurrency in Figure~\ref{lst.inversion}.a.
Furthermore, we only adapted the \code{delay} operation, but the other blocking
operations (\code{analogRead} and \code{radioWrite}) would also need to be
changed to achieve maximum concurrency.
%
Alternatively, we could resort to ISRs or implement an event-driven
scheduler to handle the operations~\cite{wsn.nesc}, but ultimately, the
program readability would still be compromised in the same way.

The program in Figure~\ref{lst.inversion}.b in \CEU extends the one in
Figure~\ref{lst.direct}.b to accommodate concurrency.
%
In contrast with the Arduino version, the original code in \CEU remains
unmodified (Figure~\ref{lst.inversion}.b, ln. 4--9) and concurrency is achieved
through the \code{par/or} construct, which creates two lines of execution and
terminates when either of them terminates, aborting the other automatically.
%
This approach preserves the sequential, easy-to-read style while introducing
concurrency seamlessly.

\subsection{Standby Considerations}
\label{sec.ceu.standby}

The structure of the program in Figure~\ref{lst.inversion}.b also indicates
which peripherals are active at a given time.
%
For instance, when the program is awaiting concurrently in lines 2 and 7,
only the radio transceiver and A/D converter can awake the program.
Hence, the language runtime can choose the most energy-efficient sleep mode
that allows these two peripherals to awake the microcontroller from associated
interrupts.
%
Since the semantics of \CEU enforces the program to always reach \code{await}
statements in all active lines of execution, it is always possible to put the
microcontroller into the optimal sleep mode after each reaction to the
environment.

\section{Standby Infrastructure}
\label{sec.standby}

In order to empower the example in Figure~\ref{lst.inversion}.b with automatic
standby, we have developed some extensions to \CEU as follows:
%
\begin{itemize}
\item We made the runtime of \CEU interrupt driven and put the microcontroller
      in standby after each reaction to the environment.
\item We provided operations for the drivers to indicate which interrupts might
      awake the program.
\item We included support for ISRs in \CEU to generate input events to the
      program and awake the microcontroller.
\end{itemize}

\begin{figure}[t]
\vspace{2mm}
\begin{lstlisting}[numbers=left]
// Exposed driver functionality

output void ADC_REQUEST; // low-level request
input  int  ADC_DONE;    // low-level response
code AnalogRead (void) -> int; // high-level abstraction

// Driver implementation

output void ADC_REQUEST do
  {
    ADMUX = 0x40 | (A0 & 0x07); // selects channel A0
    bitSet(ADCSRA, ADIE);       // enables interrupt
    bitSet(ADCSRA, ADSC);       // starts the conversion
  }
end

async/isr {ADC_vect_num} do
  { bitClear(ADCSRA, ADIE); } // disables interrupt
  var int value = {ADC}; // reads register with the value
  emit ADC_DONE(value);
end

code AnalogRead (void) -> int do
  {PM_SET(PM_ADC, 1);}
  do finalize with
      {PM_SET(PM_ADC, 0);}
  end

  emit ADC_REQUEST;
  var int value = await ADC_DONE;

  escape value;
end
\end{lstlisting}
%\rule{8.4cm}{0.37pt}
\caption{ \CEU driver for the \emph{ATmega328p} A/D converter.
\label{lst.adc}
}
\end{figure}

Figure~\ref{lst.adc} shows the driver for the A/D converter in \CEU.
This code is specific to the \emph{ATmega328p} microcontroller and must be
adapted to work in other platforms.
For simplicity, we assume in the paper that the converter has a single channel
to avoid having to deal with multiplexing.

The driver exposes raw I/O events (ln. 3--4) that will only deal with low-level
port manipulation in the microcontroller.
Output events are triggered with the \code{emit} keyword (ln. 29), while input
events are captured with the \code{await} keyword (ln. 30).
%
The output event \code{ADC\_REQUEST} actual implementation (ln. 9--15) enables
ADC interrupts and starts an analog-to-digital conversion asynchronously in the
peripheral for the single channel \code{A0}.
In \CEU, any code in between \code{\{} and \code{\}} is treated as an inline
$C$ chunk, allowing for easy integration with $C$ for low-level operations.

The \code{async/isr} construct of \CEU defines an ISR which executes
asynchronously with the program when the specified interrupt occurs.
Only ISRs can emit input events to the program.
In the example, we define an ISR to handle \code{ADC} interrupts which fire
whenever a conversion is complete (ln. 17--21).
%
Although the ISR body executes asynchronously on interrupts, the input emission
(ln. 20) only takes effect on a subsequent reaction, when the synchronous part
of the program becomes idle.
%
This way, race conditions are only possible with \code{async/isr} blocks, which
are typically hidden inside device drivers.
\CEU also provides an \code{atomic} primitive to protect critical sections of
code.

The low-level events are the pieces that vary among platforms.
A driver can also expose a higher-level portable abstraction to client code.
%
In the example, the \code{AnalogRead} abstraction (ln. 23--33) takes care of
starting and awaiting the conversion (ln. 29--30), as well as dealing with the
power management runtime (PMR).
%
The \code{PM\_SET(PM\_ADC,1)} (ln. 24) tells the system that, when entering in
sleep mode, the \code{ADC} must be kept running.
%
The \code{PM\_SET(PM\_ADC,0)} inside the \code{finalize} clause (ln. 25--27)
releases the \code{ADC} subsystem from the PMR.

The \code{finalize} construct of \CEU executes the nested code whenever its
enclosing block terminates or is aborted externally.
%
The example of Figure~\ref{lst.inversion}.b invokes the \code{AnalogRead}
abstraction (ln. 7) concurrently with \code{RadioAvail} (ln. 2).
%
The \code{AnalogRead} may terminate normally or a radio message may arrive
during the A/D conversion, causing the \code{AnalogRead} to abort abruptly.
In either case, the \code{finalize} clause executes and puts the PMR in a
consistent state.

The PMR also expects a platform-specific power management module to be able to
put the microcontroller into the most efficient sleep mode possible.
%
The code in Figure~\ref{lst.pm} implements the \code{pm\_sleep} function for
the \emph{ATmega328p} microcontroller which the PMR calls when the program
becomes idle.
%
Each device has an associated index (ln. 6--10) in the \code{pm} bit vector
(ln. 4).
%
The driver manipulates its device's index to indicate its state
(Figure~\ref{lst.adc}, ln. 24,26).
%
The \code{pm\_sleep} queries the vector to choose the appropriate sleep mode.
In the example, if the timer is active (ln. 13), the microcontroller can only
use the least efficient mode%
\footnote{
    We use an external library for the sleep modes:
    \url{http://www.rocketscream.com/blog/2011/07/04/lightweight-low-power-arduino-library/}
}
(ln. 14).
%
In the best case, e.g., if only external interrupts are required, the
microcontroller can use the most efficient mode (ln. 18).

With all the standby infrastructure set, the unmodified program of
Figure~\ref{lst.inversion}.b automatically takes advantage of the deepest
sleep modes for the longest periods of time possible.

\begin{figure}[t]
\vspace{2mm}
\begin{lstlisting}[numbers=left]
#define PM_GET(dev)   bitRead(pm,dev)
#define PM_SET(dev,v) bitWrite(pm,dev,v)

static u32 pm = 0; // up to 32 peripherals

enum {
  CEU_PM_ADC = 0,
  CEU_PM_TIMER1,
  <...>,
};

void pm_sleep (void) {
  if (PM_GET(PM_TIMER1) || <...>) {
      LowPower.idle(PM_GET(PM_ADC),<...>)
    } else if (PM_GET(PM_ADC)) {
      LowPower.adcNoiseReduction(<...>);
    } else {
      LowPower.powerDown(<...>);
    }
  }
}
\end{lstlisting}
%\rule{8.4cm}{0.37pt}
\caption{ Power management module for the \emph{ATmega328p} microcontroller.
\label{lst.pm}
}
\end{figure}

\section{Discussion}

The application of Figure~\ref{lst.inversion}.b relies solely on the driver of
Figure~\ref{lst.adc} to achieve transparent standby.
%
In \CEU, the burden to deal with standby is transferred to the device drivers,
which are write-once code written by specialists and distributed with an SDK.
By transferring the work from the applications to the language level, novice or
domain programmers never have to deal with standby explicitly.
%
In contrast, general-purpose languages typically provide low-power libraries to
deal with standby.
However, programmers still have to call these libraries explicitly,
characterizing a mechanism that is manual and error prone.

In Figure~\ref{lst.inversion}.b, when introducing concurrency, the structure of
the program remains sequential and amenable to inference of the appropriate
sleep mode.
In comparison with Arduino, whose main goal is to lower the entry barrier for
embedded development, \CEU also preserves the sequential structure for
concurrent applications.

The synchronous model of \CEU provides logical parallelism to enable proper
separation of concerns, while avoiding the hassle of explicit synchronization
primitives (e.g., locks and mutexes).
%
Yet, asynchronous interrupts provide real-time responsiveness for
time-sensitive operations closer to the hardware.

\section{Conclusion and Future Work}
\label{sec.conclusion}

In this work, we address standby efficiency for embedded devices at the level
of programming languages.
%
We propose a software infrastructure for the programming language \CEU that
encompasses a power management runtime and support for interrupt service
routines in the language.
%
Our approach relies on the synchronous semantics of the language which
guarantees that reactions to the environment always reach an idle state
amenable to standby.
%
This way, application written in \CEU can take advantage of the longest periods
of time and deepest sleep modes possible without extra programming efforts.

In future work, we will evaluate the consumption of realistic applications.
%in order to evaluate the gains in energy efficiency with the
%proposed infrastructure,
%
The Arduino community has an abundance of open-source projects which can be
rewritten in \CEU to take advantage of transparent standby.
%
In this scenario, we can evaluate the time to rewrite, the resulting program
structure, and the actual energy efficiency.

\begin{acks}
The authors would like to thank Guilherme Simas for the initial explorations
with the language extensions and development of some of the device drivers.
This work was supported by the Serrapilheira Institute (grant number
Serra-1708-15612).
\end{acks}

\balance
