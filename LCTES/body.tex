\section{Introduction}
\label{sec.introduction}

According to the International Energy Agency (IEA)~\cite{iea.data}, there were
around 14 billion traditional network-connected devices in 2013.
% (e.g., mobile phones and smart TVs).
This number is expected to increase to 50 billion by 2020 with
the proliferation of IoT devices. % (e.g., smart bulbs and fitness wearables).
%
%IoT and traditional network-connected devices already outnumber people on the
%planet by a factor of two, and the amount of data traffic is expected to grow
%at an exponential rate in the next years.
%By 2050, the energy demands of networked devices should exceed the current
%consumption of Germany and Canada combined, corresponding to 6\% of current
%global electricity consumption.
%
However, most of the energy to power these devices is consumed while they are
in \emph{standby mode}, i.e., when their software is neither transmitting or
processing data.
%
For instance, standby power accounts for 10--15\% of residential electricity
consumption, and $CO_2$ emissions related to standby are equivalent to those of
1 million cars~\cite{iea.data,standby.australia}.
%
%This undesirable inefficiency opens opportunities for improvements in energy
%savings closer to 65\% only considering currently available technologies.
%
The projected growth of IoT devices together with the surprising effects of
standby consumption, made network standby efficiency one of the six
pillars of G20's \emph{Energy Efficiency Action Plan}%
\footnote{G20's Energy Efficiency Action Plan: \url{https://www.iea-4e.org/projects/g20}}.

In this work

Synchronous language


%
\begin{itemize}
    \item To ensure that devices employ the lowest possible modes of standby consumption.
    \item To ensure that devices remain in longest possible periods of standby time.
\end{itemize}

Given the projected scale of the IoT and the role of low-power standby towards
energy efficiency, this research project has the following goals:

\begin{enumerate}
    \item Address energy efficiency through extensive use of standby.
    \item Target constrained embedded architectures that form the IoT.
    \item Provide standby mechanisms at the programming language level that scale to all applications.
    \item Support transparent/non-intrusive standby mechanisms that reduce barriers of adoption.
\end{enumerate}

% TODO: goal 1 and 3 are also the limitations: only standby, not holistic energy savings

This proposal lies at the bottom of the software development
layers---transparent programming language mechanisms---meaning that \emph{all}
applications would take advantage of low-power standby modes automatically,
without extra programming efforts.
%
We expect that existing energy-unaware applications will benefit from savings
in the order of 50\% based on IEA's estimates considering current standby
technologies~\cite{iea.data}, and previous third-party work on transparent energy
awareness~\cite{wsn.tos.2}.

\begin{acks}
  The authors would like to thank Dr. Yuhua Li for providing the
  MATLAB code of the \textit{BEPS} method.

  The authors would also like to thank the anonymous referees for
  their valuable comments and helpful suggestions. The work is
  supported by the \grantsponsor{GS501100001809}{National Natural
    Science Foundation of
    China}{http://dx.doi.org/10.13039/501100001809} under Grant
  No.:~\grantnum{GS501100001809}{61273304}
  and~\grantnum[http://www.nnsf.cn/youngscientists]{GS501100001809}{Young
    Scientists' Support Program}.

\end{acks}
