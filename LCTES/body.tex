\section{Introduction}
\label{sec.introduction}

According to the International Energy Agency (IEA), the number of
network-connected devices is expected to reach to 50 billion by 2020 with
the expansion of the Internet of Things (IoT).~\cite{iea.data}
%
However, most of the energy to power these devices will be consumed in
\emph{standby mode}, i.e., when they are neither transmitting or processing
data.
%
For instance, standby power currently accounts for 10--15\% of residential
electricity consumption, and $CO_2$ emissions related to standby are equivalent
to those of 1 million cars~\cite{iea.data,standby.australia}.
%
The projected growth of IoT devices, together with the surprising effects of
standby consumption, made network standby efficiency one of the six
pillars of G20's \emph{Energy Efficiency Action Plan}%
\footnote{G20's Energy Efficiency Action Plan: \url{https://www.iea-4e.org/projects/g20}}.

Given the projected scale of the IoT and the role of low-power standby towards
energy efficiency, this paper has the following goals:

\begin{enumerate}
\item Address energy efficiency through extensive use of standby.
\item Target low-power, resource-constrained embedded architectures that form
      the IoT.
\item Provide standby mechanisms at the programming language level that scale
      to all applications.
\item Support transparent/non-intrusive standby mechanisms that reduce barriers
      of adoption.
\end{enumerate}

Our approach lies at the bottom of the software development
layers---transparent programming language mechanisms---meaning that \emph{all}
applications take advantage of low-power standby modes automatically, without
extra programming efforts.
%
We extend the synchronous programming language
\CEU~\cite{ceu.sensys13,ceu.tecs17} with interrupt service routines (ISRs) and
a simple power management runtime (PMR).
%
Each supported microcontroller requires bindings in C for the ISRs and PMR, and
each peripheral requires a driver in \CEU.
This is a one-time process and is typically packaged and distributed in a SDK.
%
All new applications built on top of these drivers will take advantage of
standby automatically.
%
As a proof of concept, we support the 8-bit \emph{AVR/ATmega} and 32-bit
\emph{ARM/Cortex-M0} microcontrollers, and a variety of peripherals (e.g., ADC,
SPI, USART, nRF24L01 transceiver).
The project is open-source.%
\footnote{https://github.com/fsantanna/ceu-arduino/}

 Thank you, I will improve the page.

In summary:

Reactive: code executes in reactions to events

Synchronous: reactions run to completion, i.e., there's no implicit preemption or real parallelism (this avoids explicit synchronization: locks, queues, etc)

Structured: programs use structured control mechanisms, such as "await" (to suspend a line of execution), and "par" (to combine multiple awaiting lines of execution)

Structured programming avoids deep nesting of callbacks letting you write programs in direct/sequential style. In addition, when a line of execution is aborted, all allocated resources are safely released.

In comparison to FRP/dataflow, it is more imperative supporting sequences/loops/conditionals/parallels. The notion of (multiple) program counter is explicit. Also, everything is lexically scoped, there's no GC involved.

In comparison to promises/futures, it provides lexical parallel constructs, allowing the branches to share local variables and, more importantly, supporting safe abortion of code (with the "par/or").




- application on IoT
- gains??


- structured programming
- lexical scope
- automatic power management: standby as well as enabled/disabled, powered, switched

needs a requires a energy-aware driver 

process for ISR bindings and
A infra-structure programmer writes device drivers for
Each peripheral requires a driver
An architecture
Infra
Once a driver is written, all applications 

- synchronous languages / \CEU

- except drivers, which are also written in Ceu

I have been working in the design of \CEU, a new reactive programming language
targeting resource-constrained embedded systems, for the past 8
years~\cite{ceu.sensys11,ceu.tr,ceu.sensys13,ceu.rem13,ceu.phd,ceu.rebls14,
ceu.mod15,ceu.rebls15,ceu.terra,ceu.media.webmedia16,ceu.tecs17}.
%
\CEU is grounded on the synchronous concurrency model, which has been
successfully adopted in the context of hard real-time systems such as avionics
and automobiles industry since the 80's~\cite{rp.twelve}.
%
The synchronous model trades power for reliability and has a simpler model
of time that suits most requirements of IoT applications.
%
On the one hand, this model cannot directly express time-consuming
computations, such as compression and cryptography algorithms, which are
typically either absent or delegated to auxiliary chips in the context of the
IoT.
%
On the other hand, all reactions to the external world are guaranteed to be
computed in bounded
time, ensuring that applications always reach an idle state amenable to standby
mode.

- results
    - atmega 328p/2560
    - arm cortex-m0
    - how much efficiency?
%
We expect that existing energy-unaware applications will benefit from savings
in the order of 50\% based on IEA's estimates considering current standby
technologies~\cite{iea.data}, and previous third-party work on transparent energy
awareness~\cite{wsn.tos.2}.

- limitations

- paper structure

In Section~\ref{sec.ceu}, we introduce the synchronous model together with the
programming language \CEU.
In Section~\ref{sec.ext}, we present the language extensions that support
transparent standby.
In Section~\ref{sec.eval}, we evaluate .

\section{The Synchronous Language CEU}
\label{sec.ceu}

\section{Transparent Standby Mechanisms}
\label{sec.ceu}

\begin{acks}
  The authors would like to thank Dr. Yuhua Li for providing the
  MATLAB code of the \textit{BEPS} method.

  The authors would also like to thank the anonymous referees for
  their valuable comments and helpful suggestions. The work is
  supported by the \grantsponsor{GS501100001809}{National Natural
    Science Foundation of
    China}{http://dx.doi.org/10.13039/501100001809} under Grant
  No.:~\grantnum{GS501100001809}{61273304}
  and~\grantnum[http://www.nnsf.cn/youngscientists]{GS501100001809}{Young
    Scientists' Support Program}.

\end{acks}
