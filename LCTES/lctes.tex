\documentclass[sigplan, anonymous, review]{acmart}

\usepackage{booktabs} % For formal tables
\usepackage{xspace}

\usepackage{listings}
\lstdefinelanguage{ceu}{%
  language=C,
  morekeywords={%
    and, uint32_t, code, async, isr,
    await, bool, call, class, data, define, deterministic, do, each, else,
    emit, end, escape, event, every, finalize, hor, implementation, in,
    input, interface, loop, min, native, new, nohold, not, or, output, par,
    pool, pure, return, signal, spawn, tag, then, this, traverse, until,
    var, watching, when, with},
}
\lstset{
  basicstyle=\ttfamily,
  captionpos=b,
  columns=flexible,
  commentstyle=\rmfamily\itshape,
  escapeinside={||},
  escapeinside={|},
  frame=tb,
  keepspaces=true,
  keywordstyle=\bfseries,
  language=ceu,
  mathescape=true,
  numbersep=4pt,
  numberstyle=\scriptsize,
  upquote=true,
}

\newcommand{\CEU}{\textsc{C\'{e}u}\xspace}
\newcommand{\code}[1] {{\small{\texttt{#1}}}}

% Copyright
%\setcopyright{none}
%\setcopyright{acmcopyright}
%\setcopyright{acmlicensed}
\setcopyright{rightsretained}
%\setcopyright{usgov}
%\setcopyright{usgovmixed}
%\setcopyright{cagov}
%\setcopyright{cagovmixed}

% DOI
\acmDOI{10.475/123_4}

% ISBN
\acmISBN{123-4567-24-567/08/06}

%Conference
%\acmConference[WOODSTOCK'97]{ACM Woodstock conference}{July 1997}{El Paso, Texas USA}
\acmConference[LCTES'18]{ACM SIGPLAN/SIGBED}{June 2018}{Philadelphia, USA}
%\acmYear{1997}
%\copyrightyear{2016}

%\acmPrice{15.00}

%\acmBadgeL[http://ctuning.org/ae/ppopp2016.html]{ae-logo}
%\acmBadgeR[http://ctuning.org/ae/ppopp2016.html]{ae-logo}

\begin{document}
%\title{A Language-Based Approach for Standby in Embedded Systems}
\title{WIP: Transparent Standby for Low-Power, Resource-Constrained Embedded Systems}
\subtitle{A Programming Language-Based Approach}

%\titlenote{Produces the permission block, and copyright information}
%\subtitlenote{The full version of the author's guide is available as \texttt{acmart.pdf} document}

\author{Ben Trovato}
\authornote{Dr.~Trovato insisted his name be first.}
\orcid{1234-5678-9012}
\affiliation{%
  \institution{Institute for Clarity in Documentation}
  \streetaddress{P.O. Box 1212}
  \city{Dublin}
  \state{Ohio}
  \postcode{43017-6221}
}
\email{trovato@corporation.com}

\author{G.K.M. Tobin}
\authornote{The secretary disavows any knowledge of this author's actions.}
\affiliation{%
  \institution{Institute for Clarity in Documentation}
  \streetaddress{P.O. Box 1212}
  \city{Dublin}
  \state{Ohio}
  \postcode{43017-6221}
}
\email{webmaster@marysville-ohio.com}

\author{Lars Th{\o}rv{\"a}ld}
\authornote{This author is the
  one who did all the really hard work.}
\affiliation{%
  \institution{The Th{\o}rv{\"a}ld Group}
  \streetaddress{1 Th{\o}rv{\"a}ld Circle}
  \city{Hekla}
  \country{Iceland}}
\email{larst@affiliation.org}

\author{Valerie B\'eranger}
\affiliation{%
  \institution{Inria Paris-Rocquencourt}
  \city{Rocquencourt}
  \country{France}
}
\author{Aparna Patel}
\affiliation{%
 \institution{Rajiv Gandhi University}
 \streetaddress{Rono-Hills}
 \city{Doimukh}
 \state{Arunachal Pradesh}
 \country{India}}
\author{Huifen Chan}
\affiliation{%
  \institution{Tsinghua University}
  \streetaddress{30 Shuangqing Rd}
  \city{Haidian Qu}
  \state{Beijing Shi}
  \country{China}}

\author{Charles Palmer}
\affiliation{%
  \institution{Palmer Research Laboratories}
  \streetaddress{8600 Datapoint Drive}
  \city{San Antonio}
  \state{Texas}
  \postcode{78229}}
\email{cpalmer@prl.com}

\author{John Smith}
\affiliation{\institution{The Th{\o}rv{\"a}ld Group}}
\email{jsmith@affiliation.org}

\author{Julius P.~Kumquat}
\affiliation{\institution{The Kumquat Consortium}}
\email{jpkumquat@consortium.net}


% The default list of authors is too long for headers.
\renewcommand{\shortauthors}{B. Trovato et al.}

\begin{abstract}
Standby efficiency for connected devices is one of the priorities of the
\emph{G20's Energy Efficiency Action Plan}.
%
We propose transparent programming language mechanisms to enforce that
applications remain in the deepest standby modes for the longest periods of
time.
%
We extend the programming language \CEU with support for interrupt service
routines and with a simple power management runtime.
%
We also provide device drivers based on these primitives on top of which
applications can be built to take advantage of standby automatically.
%
Our approach relies on the synchronous semantics of the language which enforces
that reactions to the environment always reach an idle state amenable to
standby.
%
In addition, we show that programs in \CEU can keep a sequential structure,
even when applications require non-trivial concurrent behavior, to lower the
barrier of adoption.
\end{abstract}

%
% The code below should be generated by the tool at
% http://dl.acm.org/ccs.cfm
% Please copy and paste the code instead of the example below.
%
\begin{CCSXML}
<ccs2012>
    <concept>
        <concept_id>10010520.10010553.10010562.10010564</concept_id>
        <concept_desc>Computer systems organization~Embedded software</concept_desc>
    <concept_significance>500</concept_significance>
    </concept>
    <concept>
        <concept_id>10011007.10011006.10011041.10011048</concept_id>
        <concept_desc>Software and its engineering~Runtime environments</concept_desc>
        <concept_significance>300</concept_significance>
    </concept>
</ccs2012>
\end{CCSXML}

\ccsdesc[500]{Computer systems organization~Embedded software}
\ccsdesc[300]{Software and its engineering~Runtime environments}

\keywords{Arduino, Concurrency, Embedded Systems, Esterel, IoT, Standby}

%\begin{teaserfigure}
  %\includegraphics[width=\textwidth]{sampleteaser}
  %\caption{This is a teaser}
  %\label{fig:teaser}
%\end{teaserfigure}

\maketitle

\section{Introduction}
\label{sec.introduction}

According to the International Energy Agency (IEA)~\cite{iea.data}, there were
around 14 billion traditional network-connected devices in 2013.
% (e.g., mobile phones and smart TVs).
This number is expected to increase to 50 billion by 2020 with
the proliferation of IoT devices. % (e.g., smart bulbs and fitness wearables).
%
%IoT and traditional network-connected devices already outnumber people on the
%planet by a factor of two, and the amount of data traffic is expected to grow
%at an exponential rate in the next years.
%By 2050, the energy demands of networked devices should exceed the current
%consumption of Germany and Canada combined, corresponding to 6\% of current
%global electricity consumption.
%
However, most of the energy to power these devices is consumed while they are
in \emph{standby mode}, i.e., when their software is neither transmitting or
processing data.
%
For instance, standby power accounts for 10--15\% of residential electricity
consumption, and $CO_2$ emissions related to standby are equivalent to those of
1 million cars~\cite{iea.data,standby.australia}.
%
%This undesirable inefficiency opens opportunities for improvements in energy
%savings closer to 65\% only considering currently available technologies.
%
The projected growth of IoT devices together with the surprising effects of
standby consumption, made network standby efficiency one of the six
pillars of G20's \emph{Energy Efficiency Action Plan}%
\footnote{G20's Energy Efficiency Action Plan: \url{https://www.iea-4e.org/projects/g20}}.

In this work

Synchronous language


%
\begin{itemize}
    \item To ensure that devices employ the lowest possible modes of standby consumption.
    \item To ensure that devices remain in longest possible periods of standby time.
\end{itemize}

Given the projected scale of the IoT and the role of low-power standby towards
energy efficiency, this research project has the following goals:

\begin{enumerate}
    \item Address energy efficiency through extensive use of standby.
    \item Target constrained embedded architectures that form the IoT.
    \item Provide standby mechanisms at the programming language level that scale to all applications.
    \item Support transparent/non-intrusive standby mechanisms that reduce barriers of adoption.
\end{enumerate}

% TODO: goal 1 and 3 are also the limitations: only standby, not holistic energy savings

This proposal lies at the bottom of the software development
layers---transparent programming language mechanisms---meaning that \emph{all}
applications would take advantage of low-power standby modes automatically,
without extra programming efforts.
%
We expect that existing energy-unaware applications will benefit from savings
in the order of 50\% based on IEA's estimates considering current standby
technologies~\cite{iea.data}, and previous third-party work on transparent energy
awareness~\cite{wsn.tos.2}.

\begin{acks}
  The authors would like to thank Dr. Yuhua Li for providing the
  MATLAB code of the \textit{BEPS} method.

  The authors would also like to thank the anonymous referees for
  their valuable comments and helpful suggestions. The work is
  supported by the \grantsponsor{GS501100001809}{National Natural
    Science Foundation of
    China}{http://dx.doi.org/10.13039/501100001809} under Grant
  No.:~\grantnum{GS501100001809}{61273304}
  and~\grantnum[http://www.nnsf.cn/youngscientists]{GS501100001809}{Young
    Scientists' Support Program}.

\end{acks}


\bibliographystyle{ACM-Reference-Format}
\bibliography{my,standby,other}

\end{document}
